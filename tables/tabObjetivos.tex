% Objetivos
\vspace{1cm}
\label{tab-objs}
\begin{tabular}{|p{.4\textwidth}|p{.3\textwidth}|p{.12\textwidth}|}

  \tabheadformat
  \tabhead{Objetivo}         &
  \tabhead{Referencia}       &
  \tabhead{Conseguido}  \\
\hline
\textbf{O1. Representación del contexto mediante reglas} & Sección
\ref{iter6-reasoner} & \checkmark \\
\hline

\textbf{O2. Elección o desarrollo de un vocabulario para resultados} & Anexo \ref{chap:ontologia} & \checkmark \\
\hline

\textbf{O3. Diseño e implementación de primitivas de evaluación} & Sección
\ref{sec:dqassessment-completeness}, \ref{iteracion5} y \ref{iteracion6} & \checkmark \\
\hline

\textbf{O4. Elección o implementación de un razonador de reglas} & Sección
\ref{iter6-reasoner} & \checkmark \\
\hline

\textbf{O5. Implementación de las primitivas en entorno Big Data} & Sección
\ref{iteracion4}, \ref{iteracion5} y \ref{iteracion6} & \checkmark \\
\hline

\textbf{O6. Desarrollo de una aplicación de prueba de concepto} & Sección
\ref{iteracion7} & \checkmark \\
\hline

\end{tabular}
\captionof{table}{Consecución de objetivos}


% Local variables:
%   coding: utf-8
%   ispell-local-dictionary: "castellano8"
%   TeX-master: "main.tex"
% End:
