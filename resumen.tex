\chapter{Resumen}


La Web Semántica es un tipo de web cuyo contenido puede ser procesado tanto por
máquinas de forma automática como por personas. Esto es posible debido a una serie de avances significativos en
la representación del conocimiento durante los últimos años. Esta
representación del conocimiento se basa en el concepto de \textit{triplas} (sujeto,
predicado, objeto) que establecen relaciones entre los distintos datos. Estas
relaciones a su vez se representan utilizando lenguajes y especificaciones
desarrollados para tal fin, como son \acf{RDF} y \acf{OWL}. 

Paralelamente surge el paradigma de Linked Data, que se puede definir como el
uso del modelo web para publicar datos estructurados de manera que puedan ser
fácilmente consumidos y enlazados con otros. Aprovechando la capacidad
semántica obtenida mediante la representación en triplas y la capacidad de poder
interconectar los datos de esta manera, se pueden tener volúmenes significativos de
datos interrelacionados sobre los cuales aplicar razonamiento automático,
consultas o cualquier otra operación como parte del procesamiento automático
anteriormente citado. 

Por otra parte, tener en cuenta la calidad de los datos resulta algo
imprescindible en el desarrollo con éxito de cualquier tarea. Se puede encontrar
una gran cantidad de \textit{frameworks} (Jena, Sesame, \ldots) con los que elaborar
aplicaciones semánticas, pero actualmente ninguno de ellos
incorpora primitivas para dar soporte a las operaciones de gestión de calidad de
datos que están siendo utilizados. Dada la importancia de la calidad de datos y
la carencia de dichas utilidades en los frameworks anteriormente citados, es
patente la necesidad de elaborar mecanismos que permitan llevar a cabo evaluaciones de
calidad de los datos usados en aplicaciones de tecnología semántica en general y
Web Semántica en particular.

Así pues el objetivo del Proyecto Fin de Carrera consistió en la elaboración de una extensión de un framework de desarrollo
de aplicaciones de Web Semántica y Linked Data: Apache Jena. La finalidad de
dicha extensión fue, dado un conjunto de Linked Data, dar soporte a primitivas
de medición de calidad de datos teniendo en cuenta el contexto en el que dichos
datos deben ser considerados.

El objetivo principal de una posible iteración sobre este proyecto fue extender
estas funcionalidades en el contexto Big Data, debido a la falta de eficiencia
encontrada cuando las métricas calculadas ganaban complejidad e involucraban un
número considerable de triplas.

De esta manera, el objetivo principal de este Trabajo Fin de Máster es el
desarrollo de un stack para un framework de procesamiento de datos en entornos
distribuidos, Apache Spark, de manera que se pueda suplir la necesidad de
evaluación de calidad de datos enlazados en entornos Big Data. 


\chapter{Abstract}


The Semantic Web is a type of Web which content can be processed
automatically. This is possible due to a number of significant advances in
knowledge representation in recent years. Knowledge in the Semantic Web is
represented by triples (subject, predicate, object) that establish relations
between different data using languages and specifications developed for this
purpose, such as Resource Definition Language (\acs{RDF}) and Ontology Web Language
(\acs{OWL}).

Meanwhile, Linked Data paradigm appears, which can be defined as the use of the
Web model to publish structured data so it can be easiliy consumed and linked to
other data. Considering the semantic capacity obtained by triples representation
and the ability to interconnect data thus, users could have significant volumes
of interrelated data where apply automated reasoning, inquiries or any other
operation.

Moreover, it should be considered data quality as a need in the successful
perform of any task. People can find a lot of frameworks (Jena, Sesame, . . . )
to develop semantic application, but currently none of them incorporates basic
operations for assessing quality of data in use. Many authors consider that this
part has not yet been taken int account or has been received little
attention. Therefore, there is a need to develop mechanisms to carry out quality
assessments of the data used in semantic applications.

Thus, the objective of the original project involved the development of an
extension for a Semantic Web and Linked Data framework: Apache Jena. The purpose
of this extension was, having a set of Liked Data, to support a set of data
quality measures taking into account the context in which these data should be
considered.

The main target of a possible next iteration over the project was to extend this
functionalities on a Big Data context, due to lack of efficiency found when the
metrics calculation get complex and involves an increasing number of triples.

Thus, the goal of this project is the development of a stack for a
distributed processing framework, Apache Spark, in order to cover the need of
Data Quality Assessment of Linked Data in a Big Data environment. 

