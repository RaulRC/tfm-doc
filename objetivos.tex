\chapter{Objetivos}
\label{chap:objetivos}

\section{Objetivos principales}

Los objetivos de este trabajo fin de máster (\acf{TFM}) son:

\begin{itemize}
\item Desarrollar una extensión para el framework XX con el fin de dar soporte a primitivas de evaluación de
  calidad de datos en un entorno Big Data. Dicha extensión se generará como una \acs{API} cuya finalidad será encapsular todas
las primitivas desarrolladas y facilitar al usuario la utilización de dicha
extensión.

\begin{itemize}
\item Establecer un modelo de
  desarrollo distribuido para uno de estos frameworks con el fin de dar soporte
  a primitivas de evaluación de calidad de datos en entorno Big Data.
\end{itemize}

\item Propuesta de arquitectura Big Data orientada al manejo y evaluación de
  datos semánticos.
\end{itemize}

\section{Objetivos parciales}

El objetivo principal implica el abordar una serie de objetivos parciales
relacionados con la arquitectura y funcionalidades del sistema que se pretenden
desarrollar. Dichos objetivos parciales se detallan a continuación. 


\begin{definitionlist}
%1
\item[O1. Representación del contexto de evaluación de calidad de datos]
Implementar  las  reglas  de  negocio  que representen el contexto necesario
para la evaluación de
calidad  de datos usando la tecnología más adecuadas.
\label{sec:reglas}

%2
\item[O2. Desarrollo de un vocabulario para los resultados
  de evaluaciones de calidad de datos]
El fin de dicho vocabulario será el de permitir publicar los
resultados de las evaluaciones que se lleven a cabo de manera formal y basándose
en los paradigmas de la Web Semántica. Los conceptos de ontología y vocabulario
serán detallados en la Sección \ref{sbs:ontologia}. 

%3
\item[O3. Diseño e implementación de las primitivas de evaluación de calidad de datos]

Diseñar e implementar primitivas para la evaluación de calidad de los datos como una
extensión del framework elegido, considerando el contexto de uso mediante el uso
de reglas semánticas.

Estas primitivas se desarrollarán tomando diferentes perspectivas de la calidad
de los datos, que se pueden consultar en la Sección \ref{sec:dqREF}. 


%4
\item[O4. Elección de un razonador de reglas para datos semánticos]

Se realizará un estudio sobre los razonadores de reglas de inferencia para
datos semánticos. Las reglas definidas deberán ser
ejecutadas por un razonador que generará los resultados finales del proceso de
medición. 


%5
\item[O5. Desarrollo de una aplicación de prueba de concepto]
Elaborar un pipeline que haga uso de la extensión
previamente desarrollada. Para ello se deberá implementar la arquitectura de
la aplicación teniendo en cuenta las restricciones de la tecnología a utilizar.

\end{definitionlist}
