\chapter{Resultados}
\label{chap:resultados}

\drop{E}{n} este capítulo se presentan los resultados obtenidos tras llevar a
cabo el plan de trabajo presentado en el capítulo anterior. El proceso ha
desembocado en la obtención de varios artefactos, principalmente SparkDQ y su
prueba de concepto, tal y como se pretendía desarrollar. 

\section{Fase de Inicio}

Durante la fase inicial, se destina una iteración a planificar el desarrollo del
\acs{TFM}, establecer requisitos y en función de éstos extraer una serie de
casos de uso.

Las tablas expuestas en el capítulo anterior se van a exponer de nuevo con el
fin de facilitar la lectura.

\subsection{Iteración 1: Planificación, casos de uso y requisitos}

% Iteración 1
\vspace{1cm}
\begin{tabular}{|p{.4\textwidth}|p{.4\textwidth}|}

\hline

\cellcolor[gray]{0.7}Fase del \acs{PUD} & Inicio 
 \\
\hline

\cellcolor[gray]{0.7}Flujo de trabajo del \acs{PUD} & Análisis y Diseño
 \\
\hline

\cellcolor[gray]{0.7}Objetivos  &
\cellcolor[gray]{0.7}Artefactos de Salida \\
\hline

\begin{itemize}
\item Contextualizar trabajo \acs{PFC}
\item \acs{CdU} iniciales y requisitos
\item Arquitectura preliminar
\end{itemize}

&

\begin{itemize}
\item Plan de actualización artefactos \acs{PFC}
\item Diagrama \acs{CdU} y \acf{RF}
\item Arquitectura preliminar
\end{itemize} \\
\hline
\end{tabular}
\captionof{table}{Iteración 1: Fase de Inicio}


% Local variables:
%   coding: utf-8
%   ispell-local-dictionary: "castellano8"
%   TeX-master: "main.tex"
% End:


\section{Fase de Elaboración}

\subsection{Iteración 2: Actualización de JenaDQ}

% Iteración 2
\vspace{1cm}
\begin{tabular}{|p{.4\textwidth}|p{.4\textwidth}|}

\hline

\cellcolor[gray]{0.7}Fase del \acs{PUD} & Elaboración
 \\
\hline

\cellcolor[gray]{0.7}Flujo de trabajo del \acs{PUD} & Análisis, Diseño, Implementación y Pruebas
 \\
\hline



\cellcolor[gray]{0.7}Objetivos  &
\cellcolor[gray]{0.7}Artefactos de Salida \\
\hline

\begin{itemize}
\item Actualización JenaDQ
\end{itemize}

&

\begin{itemize}
\item JenaDQ actualizado
\end{itemize}
 \\
\hline
\end{tabular}
\captionof{table}{Iteración 2: Actualización de JenaDQ}


% Local variables:
%   coding: utf-8
%   ispell-local-dictionary: "castellano8"
%   TeX-master: "main.tex"
% End:


\subsection{Iteración 3: Diseño de la arquitectura de SparkDQ}

% Iteración 3
\vspace{1cm}
\begin{tabular}{|p{.4\textwidth}|p{.4\textwidth}|}

\hline

\cellcolor[gray]{0.7}Fase del \acs{PUD} & Elaboración
 \\
\hline

\cellcolor[gray]{0.7}Flujo de trabajo del \acs{PUD} & Análisis y Diseño

 \\
\hline


\cellcolor[gray]{0.7}Objetivos  &
\cellcolor[gray]{0.7}Artefactos de Salida \\
\hline

\begin{itemize}
\item Diseñar nueva arquitectura de framework
\item Diserñar arquitectura PoC
\end{itemize}

&

\begin{itemize}
\item Diseño de SparkRDF y SparkDQ
\item Diseño de \acs{PoC} preliminar
\end{itemize}

 \\
\hline
\end{tabular}
\captionof{table}{Iteración 3: Diseño de soluciones }


% Local variables:
%   coding: utf-8
%   ispell-local-dictionary: "castellano8"
%   TeX-master: "main.tex"
% End:


\section{Fase de Construcción}

\subsection{Iteración 4: Desarrollo de SparkRDF}

% Iteración 4
\vspace{1cm}
\begin{tabular}{|p{.4\textwidth}|p{.4\textwidth}|}

\hline

\cellcolor[gray]{0.7}Fase del \acs{PUD} & Elaboración
 \\
\hline

\cellcolor[gray]{0.7}Flujo de trabajo del \acs{PUD} & Análisis, Diseño,
Implementación y Pruebas
 \\
\hline


\cellcolor[gray]{0.7}Objetivos  &
\cellcolor[gray]{0.7}Artefactos de Salida \\
\hline

\begin{itemize}
\item Desarrollo de SparkRDF
\item Primitivas de lectura de triplas y exportar a modelo distribuido
\item \acs{QA}
\item Documentación SparkRDF
\end{itemize}

&

\begin{itemize}
\item Artefacto SparkRDF
\item Documentación SparkRDF
\end{itemize}
\\
\hline
\end{tabular}
\captionof{table}{Iteración 4: Desarrollo de SparkRDF}


% Local variables:
%   coding: utf-8
%   ispell-local-dictionary: "castellano8"
%   TeX-master: "main.tex"
% End:


\subsection{Iteración 5: Desarrollo de SparkDQ para la métrica \textit{Interlinking}}

% Iteración 5

\begin{tabular}{|p{.4\textwidth}|p{.4\textwidth}|}

\hline

\cellcolor[gray]{0.7}Fase del \acs{PUD} & Elaboración
 \\
\hline

\cellcolor[gray]{0.7}Flujo de trabajo del \acs{PUD} & Análisis, Diseño,
Implementación y Pruebas
 \\
\hline

\cellcolor[gray]{0.7}Fechas Inicio - Fin  &
 \\
\hline

\cellcolor[gray]{0.7}Objetivos  &
\cellcolor[gray]{0.7}Artefactos de Salida \\
\hline

\begin{itemize}
\item Desarrollo de SparkDQ
\item Métrica Interlinking
\item QA
\end{itemize}

&

\begin{itemize}
\item Artefacto SparkDQ preliminar con primera métrica
\end{itemize}
\\
\hline
\end{tabular}
\captionof{table}{Iteración 5}


% Local variables:
%   coding: utf-8
%   ispell-local-dictionary: "castellano8"
%   TeX-master: "main.tex"
% End:


\subsection{Iteración 6: Desarrollo de SparkDQ para la métrica \textit{SchemaCompleteness}}

% Iteración 6
\vspace{1cm}
\begin{tabular}{|p{.4\textwidth}|p{.4\textwidth}|}

\hline

\cellcolor[gray]{0.7}Fase del \acs{PUD} & Elaboración
 \\
\hline

\cellcolor[gray]{0.7}Flujo de trabajo del \acs{PUD} & Análisis, Diseño,
Implementación y Pruebas
 \\
\hline

\cellcolor[gray]{0.7}Fechas Inicio - Fin  &
 \\
\hline

\cellcolor[gray]{0.7}Objetivos  &
\cellcolor[gray]{0.7}Artefactos de Salida \\
\hline

\begin{itemize}
\item Desarrollo de SparkDQ
\item Métrica SchemaCompleteness
\item \acs{QA}
\end{itemize}

&

\begin{itemize}
\item Artefacto SparkDQ preliminar con segunda métrica
\end{itemize}
\\
\hline
\end{tabular}
\captionof{table}{Iteración 6}


% Local variables:
%   coding: utf-8
%   ispell-local-dictionary: "castellano8"
%   TeX-master: "main.tex"
% End:


\section{Fase de Transición}

\subsection{Iteración 7: Desarrollo de la PoC}

% Iteración 7
\vspace{1cm}
\begin{tabular}{|p{.4\textwidth}|p{.4\textwidth}|}

\hline

\cellcolor[gray]{0.7}Fase del \acs{PUD} & Transición
 \\
\hline

\cellcolor[gray]{0.7}Flujo de trabajo del \acs{PUD} & Análisis, Diseño,
Implementación y Pruebas
 \\
\hline


\cellcolor[gray]{0.7}Objetivos  &
\cellcolor[gray]{0.7}Artefactos de Salida \\
\hline

\begin{itemize}
\item Desarrollo de \acs{PoC}
\item Implementación de arquitectura \acs{PoC}
\item \acs{QA}
\item Documentación \acs{PoC}
\end{itemize}

&

\begin{itemize}
\item Artefacto \acs{PoC}
\item Arquitectura final \acs{PoC}
\item Documentación \acs{PoC}
\end{itemize}
\\
\hline
\end{tabular}
\captionof{table}{Iteración 7: Prueba de Concepto}


% Local variables:
%   coding: utf-8
%   ispell-local-dictionary: "castellano8"
%   TeX-master: "main.tex"
% End:


\subsection{Iteración 8: Entrega de TFM}

% Iteración 8
\vspace{1cm}
\begin{tabular}{|p{.4\textwidth}|p{.4\textwidth}|}

\hline

\cellcolor[gray]{0.7}Fase del \acs{PUD} & Transición
 \\
\hline

\cellcolor[gray]{0.7}Flujo de trabajo del \acs{PUD} & Documentación

 \\
\hline


\cellcolor[gray]{0.7}Objetivos  &
\cellcolor[gray]{0.7}Artefactos de Salida \\
\hline

\begin{itemize}
\item Elaboración de la documentación para los distintos artefactos (SparkRDF, SparkDQ)
\item Elaboración de la documentación para la \acs{PoC}
\item Documentación \acs{TFM}
\end{itemize}

&

\begin{itemize}
\item Documentación artefactos
\item Documentación \acs{PoC}
\item Documentación \acs{TFM}
\end{itemize}
\\
\hline
\end{tabular}
\captionof{table}{Iteración 8}


% Local variables:
%   coding: utf-8
%   ispell-local-dictionary: "castellano8"
%   TeX-master: "main.tex"
% End:

