\chapter{Conclusiones}
\label{chap:conclusiones}
\drop{E}{n} este capítulo se van a exponer las conclusiones obtenidas tras el trabajo
realizado
en la elaboración de este \acs{TFM} . Incluye además un apartado de propuestas de
trabajo
futuro, donde se describen posibles trabajos adicionales tomando como base el
desarrollo de
este proyecto, con el fin de abrir nuevas líneas de investigación o profundizar
en el ámbito
en el que se enmarca.

\section{Conclusiones}
A continuación se exponen las conclusiones obtenidas sobre los distintos
objetivos parciales y una panorámica sobre el total de la elaboración del
\acs{TFM}.

\subsection{Consecución de objetivos}
Seguidamente se enumera la consecución de objetivos, que queda resumida en la
tabla \ref{tab-objs}.

% Objetivos
\vspace{1cm}
\label{tab-objs}
\begin{tabular}{|p{.4\textwidth}|p{.3\textwidth}|p{.1\textwidth}|}

  \tabheadformat
  \tabhead{Objetivo}         &
  \tabhead{Referencia}       &
  \tabhead{Conseguido}  \\
\hline
\textbf{O1. Representación del contexto mediante reglas} & Sección
\ref{iter6-reasoner} & \checkmark \\
\hline

\textbf{O2. Elección o desarrollo de un vocabulario para resultados} & Anexo \ref{chap:ontologia} & \checkmark \\
\hline

\textbf{O3. Diseño e implementación de primitivas de evaluación} & Sección
\ref{sec:dqassessment-completeness}, \ref{iteracion5} y \ref{iteracion6} & \checkmark \\
\hline

\textbf{O4. Elección o implementación de un razonador de reglas} & Sección
\ref{iter6-reasoner} & \checkmark \\
\hline

\textbf{O5. Implementación de las primitivas en entorno Big Data} & Sección
\ref{iteracion4}, \ref{iteracion5} y \ref{iteracion6} & \checkmark \\
\hline

\textbf{O6. Desarrollo de una aplicación de prueba de concepto} & Sección
\ref{iteracion7} & \checkmark \\
\hline

\end{tabular}
\captionof{table}{Consecución de objetivos}


% Local variables:
%   coding: utf-8
%   ispell-local-dictionary: "castellano8"
%   TeX-master: "main.tex"
% End:


\begin{definitionlist}
\item[O1. Representación del contexto de evaluación de calidad de datos]

Mediante el uso de reglas sencillas basadas en valores sobre las métricas,
se ha conseguido dar forma al contexto, permitiendo al
usuario indicar qué parámetros y niveles de calidad considera necesarios 
y con qué valores para el desarrollo de su tarea. Se
puede consultar al respecto en las sección \ref{iter6-reasoner}.


\item[O2. Desarrollo de un vocabulario para los resultados de evaluaciones de
  calidad de datos]

El vocabulario desarrollado, que puede consultarse en el Anexo
\ref{chap:ontologia}, es suficientemente flexible y representativo como para
ofrecer resultados de evaluaciones sencillas como de planes de evaluación. Esta
ontología fue desarrollada en el transcurso del \acs{PFC} y se ha reutilizado
para este particular. 

Tal y como se ha comentado en la sección \ref{onto-results}, se ha desacoplado
cualquier vocabulario u ontología, pudiendo ser el usuario final quien elija qué
ontología utilizar para representar sus resultados de calidad, haciendo del
stack SparkDQ una pieza más usable y versátil. 

\item[O3. Diseño e implementación de las primitivas de evaluación de calidad de
  datos]

Los algoritmos desarrollados demuestran ser eficaces y completos. Los resultados
se ajustan a las decisiones sobre la medición tomadas en este proyecto. Se
pueden consultar tanto algoritmos como todo el proceso de desarrollo en la
sección \ref{sec:dqassessment-accessibility}. 

Dichos algoritmos se han desarrollado en dos etapas:

\begin{enumerate}
\item Una primera fase fue desarrollada durante el \acs{PFC} (véase \cite{PFC})
  de manera centralizada. 
\item La segunda fase ha sido desarrollada en un contexto Big Data durante el
  presente trabajo.
\end{enumerate}

Se pueden consultar los resultados de la comparativa entre desarrollos de
primitivas en la sección \ref{comparativa}. 


\item [O4. Elección de un razonador de reglas para datos semánticos]

Los razonadores incorporados en Jena permiten añadir nuevas triplas a los
modelos de evaluación en función de las reglas que tomen como base, luego sirven
al propósito de llevar a cabo evaluaciones considerando el contexto definido en
esas reglas. 

No obstante para el problema del razonamiento en distribuido se ha concluído que
requiere de un trabajo más específico y profundo, ofreciéndose aquí una solución
basada en \acf{UDF} y un módulo extensible de razonamiento incluido en SparkDQ,
tal y como se ha tratado en la sección \ref{iter6-reasoner}. 

\item [O5. Desarrollo de una aplicación de prueba de concepto]

Se ha elaborado una \acs{PoC} obteniendo unos resultados excelentes a la hora de
llevar a cabo evaluaciones de calidad de datos en un entorno distribuido,
emulado en máquinas locales cuyas características se pueden contrastar en la
sección \ref{equipos-desarrollo}.

El resultado final de la \acs{PoC} puede consultarse en la sección \ref{iteracion7} y los resultados
comparativos con JenaDQ en la sección \ref{comparativa}. 

\end{definitionlist}

\section{Propuestas de trabajos futuros}


\begin{definitionlist}

\item[Adición de nuevas \acs{DQD}]

Incrementar el número de dimensiones de calidad sería una labor sumamente
interesante de cara a completar un Modelo de Calidad de Datos. En este proyecto
el desarrollo se ha centrado en dos de ellas, que han sido consideradas de
crucial importancia para el contexto en el que se enmarcan, pero existen otras
muchas cuya adición aportaría mucho valor a la extensión de JenaDQ y SparkDQ que se ha
realizado. 
  

\end{definitionlist}

\section{Publicaciones}
??
\section{Opinión personal}
bla
