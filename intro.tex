\chapter{Introducción}
\label{chap:Introduccion}

\section{Descripción del problema}


\drop{E}{n} \cite{bernerslee2001semantic} se define la Web Semántica como
un tipo de Web cuyo contenido resulta procesable de manera automática. Para
conseguir un contenido procesable se debe antes llevar a cabo una
representación o formalización del conocimiento que se pretende exponer. Esto ha
sido posible gracias al concepto de \textit{tripla}
que no es más que la relación, expresada en un lenguaje formal, existente entre un sujeto y un objeto a través de
un predicado que los relaciona. De esta manera, si se desea expresar formalmente una sentencia
respecto de cualquier concepto bastará con establecer un sujeto o recurso (el
propio concepto) una relación o propiedad que se desee modelar, y un objeto que
a su vez puede ser un nuevo recurso de manera que agrupando conjuntos de triplas
se logra definir conceptos completos de manera formal. 


Durante los últimos años se han desarrollado lenguajes como \acs{RDF} y
\acs{OWL}, y
especificaciones, con el fin de dar soporte a 
la interoperabilidad semántica \cite{Shadbolt:2006:SWR:1155313.1155373}. En
1997 el \acf{W3C} ~\cite{W3CSW} define la primera especificación de \acs{RDF} lo que sentará los cimientos de la Web Semántica: el
objetivo de \acs{RDF} será aportar una descripción semántica del conocimiento en la
Web \cite{Shadbolt:2006:SWR:1155313.1155373} como lenguaje de definición de
recursos, haciendo posible trabajar con la totalidad de la Web
como un conjunto de recursos y sus interrelaciones. 

Posteriormente surge el paradigma Linked Data \cite{bernerslee:2009}, que se puede definir
como el uso del modelo Web para publicar datos estructurados de manera que
puedan ser fácilmente consumidos y combinados con otros ~\cite{bizer_linked_2009}. Valiéndose del concepto
de \acf{URI}, se consiguen identifcar recursos en la red de forma unívoca y así
poder enlazarlos sin ambigüedad. Linked Data
aprovecha la potencia descriptiva de \acs{RDF} para facilitar el procesado
y razonamiento automático sobre grandes conjuntos de datos. En
\cite{bernerslee:2009} se enumera una serie de principios que deben cumplir los
datos para ser considerados Linked Data: 


\begin{enumerate}
\item{Usar \acs{URI} como identificadores de los recursos publicados en la Web.}
\item{Usar las URL de estas \acs{URI} para que la gente pueda localizar y consultar
  estos recursos.}
\item{Proporcionar información útil cuando la \acs{URI} sea desreferenciada.}
\item{Incluir vínculos a otras \acs{URI} relacionadas con los datos en el recurso.}
\end{enumerate}


Por otra parte, el concepto de \textit{Calidad} puede aplicarse a cualquier tarea u objeto. Es
importante matizar para el contexto del proyecto qué se entiende por calidad y
más concretamente por calidad de los datos (\acs{DQ}). 


Existen varias definiciones de calidad que conviene considerar a la hora de
abordar el proyecto. En primer lugar, se va a exponer la definición que la Real
Academia Española de la lengua otorga a este término: 

\textit{Propiedad o conjunto de propiedades
  inherentes a algo que permiten juzgar su valor}.

Sin embargo este concepto ha ido variando con el tiempo, adaptándose a las
necesidades y a los procesos. Relacionando este concepto con la
\textit{fabricación} de un determinado bien, en ~\cite{ISHI} se define la calidad
como: 

\textit{Un producto tiene calidad cuando es desarrollado, diseñado y mantenido
  de la forma más económica, útil y satisfactoria para el consumidor}. 


Autores como \cite{conf/webist/CaballeroMACC} o \cite{conf/ekaw/FurberH10} dejan constancia de cómo la calidad de los datos resulta imprescindible en el
desarrollo con éxito de cualquier tarea o toma de decisión. Un nivel inadecuado de calidad en los datos puede tener impactos sustanciales en ámbitos
sociales y económicos. Las empresas están mejorando la calidad de los datos con
enfoques y herramientas prácticas \cite{Wang:1996:BAD:1189570.1189572}. 

En el ámbito de las tecnologías semánticas, la comunidad ha prestado poca
atención a la calidad de los datos que han sido representados mediante la
utilización de tecnologías semánticas
\cite{conf/ekaw/FurberH10}. Existen en la actualidad un número considerable de
frameworks disponibles para el desarrollo de aplicaciones basadas en el uso de
tecnologías semánticas, no obstante, ninguno de ellos incluye funcionalidades
específicas que permitan procesar una evaluación del nivel de calidad de los
datos que se están usando.

Puesto que ningún framework de desarrollo de Web Semántica facilita este tipo de
operaciones de medición, existe una necesidad de elaborar
mecanismos que permitan llevar a cabo evaluaciones de calidad de los datos
usados en aplicaciones de tecnología semántica en general y Web Semántica en
particular.

Considerando la no disponibilidad de operaciones de evaluación de calidad de datos en los
frameworks de desarrollo de tecnologías semánticas, el objetivo del \acf{PFC}
consiste en el desarrollo de una
extensión para uno de estos frameworks. La finalidad de
dicha extensión será la de diseñar, implementar y poner a disposición de la
comunidad un conjunto de primitivas de evaluación de calidad de datos para uno de
los frameworks de desarrollo de aplicaciones de tecnología semántica más
importantes, considerando el contexto en el que los
datos deben son usados. 

Tras llevar a cabo un estudio acerca del estado actual de los frameworks de
desarrollo de Web Semántica y Linked Data, se ha escogido uno de ellos basado en
Java, \textbf{Apache Jena}, y se ha extendido con la finalidad de ofrecer estos
mecanismos para la evaluación de la calidad de datos semánticos. Jena se está
convirtiendo en un referente en la comunidad debido al amplio abanico
de tecnologías que implementa, que permite desde el manejo de datos semánticos
hasta su almacenamiento y publicación. 

El modo en que
se ha diseñado dicha extensión facilitaría en un futuro el trabajo de reciclado
y expansión del presente trabajo.



\section{Estructura del documento}

La forma en la que se ha organizado el presente documento se expone a
continuación: 

\begin{definitionlist}
\item[Capítulo \ref{chap:Introduccion}: \nameref{chap:Introduccion}] En este
  apartado se introduce el problema y se plantea el trabajo en base a los
  objetivos fijados. 


\item[Capítulo \ref{chap:objetivos}: \nameref{chap:objetivos}] En este capítulo
  se describe el principal objetivo del \acs{PFC} así como un conjunto de
  objetivos parciales que es necesario alcanzar para cubrir el objetivo
  final. 

\item[Capítulo \ref{chap:estadoarte}: \nameref{chap:estadoarte}] Esta sección
  introducirá los conceptos más importantes tratados en el documento y el
  proyecto, acompañado de conceptos y definiciones. Incluirá referencias al
  ámbito de la calidad de los datos, la Web Semántica y las tecnologías
  asociadas a ésta. 

\item[Capítulo \ref{chap:metodologia}: \nameref{chap:metodologia}] Se expondrá
  detalladamente cómo se ha utilizado la metodología de desarrollo elegida para la realización del \acs{PFC}, que
  será \acf{PUD}. Se incluirá la planificación de las iteraciones y finalmente
  un resumen de todo el ecosistema tecnológico utilizado para el desarrollo.

\item[Capítulo \ref{chap:resultados}: \nameref{chap:resultados}] En este
  capítulo se expondrán los resultados obtenidas durante las diferentes
  iteraciones del \acs{PUD} que se han llevado a cabo durante el desarrollo del
  \acs{PFC}. 

\item[Capítulo \ref{chap:conclusiones}: \nameref{chap:conclusiones}] En este
  capítulo se expondrán las conclusiones que se han obtenido tras la realización
  de este \acs{PFC} además de una serie de propuestas de trabajo futuro. 

\item[Anexo \ref{chap:manualusuario}: \nameref{chap:manualusuario}] Este anexo
  se corresponde con el manual de usuario de la aplicación de prueba de
  concepto, LiDQA Tool. 

\item[Anexo \ref{chap:instalacion}: \nameref{chap:instalacion}] Este anexo
  se corresponde con el manual de instalación de la aplicación de prueba de
  concepto, LiDQA Tool. 

\item[Anexo \ref{chap:ontologia}: \nameref{chap:ontologia}] Se
  corresponde con el listado de código del vocabulario creado para formalizar
  los resultados de evaluaciones de calidad de datos enlazados. 

\item[Anexo \ref{chap:variouscodes}: \nameref{chap:variouscodes}] El último de
  los anexos muestra algunos fragmentos de código que pueden resultar de
  interés. 

\item[Contenido del \acs{DVD} adjunto] En el \acs{DVD} adjunto a este documento
  se incluyen los siguientes artefactos generados durante la elaboración del
  \acs{PFC}: 
  \begin{itemize}
  \item Documentación del \acs{PFC} en formato \acs{PDF}.
  \item Código de \textbf{JenaDQ} incluido como proyecto para Eclipse. 
  \item Código de \textbf{LiDQA Tool} incluido como proyecto para Eclipse. 
  \item Archivo \texttt{.war} preparado para despliegue de la herramienta
    LiDQA Tool: \texttt{SemanticAnalytics.war}. 
  \item Vocabulario generado para la descripción de
    evaluaciones de calidad de datos en el fichero\texttt{DQA.owl} 
  \item Vídeo de demostración de uso de LiDQA Tool. 
  \item Documentación generada para la \acs{API} de JenaDQ en formato \acs{HTML}. 
  \end{itemize}

\end{definitionlist}




