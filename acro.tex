\chapter{Listado de acrónimos}
%alfabético
{\small
\begin{acronym}[XXXXXXXX]
  \acro{AAA}     {\textit{Anyone can say Anything about Any topic}}
  \acro{API}     {Application Programming Interface}

  \acro{CASE}    {Computer Aided Software Engineering}
  \acro{CdU}     {Caso de Uso}
  \acro{CSS}     {Cascade Style Sheet}
  \acro{CSV}     {Comma Separated Values}

  \acro{DQ}      {Calidad de Datos}
  \acro{DQD}     {Dimensión de Calidad de Datos}
  \acro{DVD}     {Digital Versatile Disc}
  \acro{DOAP}    {Description Of A Project}

  \acro{ER}      {Entidad-Interrelación}

  \acro{FOAF}    {Friend Of A Friend}

  \Acro{GNU}     {\acs{GNU} is Not Unix}

  \acro{HTML}    {HyperText Markup Language}
  \acro{HTTP}    {HyperText Transfer Protocol}

  \acro{IDE}     {Integrated Development Environment}
  \acro{ISO}     {International Organization for Standarization}

  \acro{JRDF}    {Java for \acs{RDF}}
  \acro{JSP}     {Java Server Pages}

  \acro{LD}      {Datos Enlazados}
  \acro{LOD}     {Datos Enlazados Abiertos}

  \acro{OD}      {Datos Abiertos}
  \acro{OO}      {Orientación a Objetos}
  \acro{OWL}     {Lenguaje de Ontología Web}

  \acro{PDF}     {Portable Document Format}
  \acro{PFC}     {Proyecto Fin de Carrera}
  \acro{PUD}     {Proceso Unificado de Desarrollo}

  \acro{QA}      {Quality Assurance}

  \acro{RAE}     {Real Academia Española de la Lengua}
  \acro{RDD}     {Resilient Distributed Dataset}
  \acro{RDF}     {Framework de Descripción de Recursos}
  \acro{RF}      {Requisitos Funcionales}
  \acro{RNF}     {Requisitos No Funcionales}
  \acro{RQL}     {Relationship Query Language}
  \acro{RPC}     {Remote Procedure Call}

  \acro{SGML}    {Standard Generalized Markup Language}
  \acro{SKOS}    {Simple Knowledge Organization System}
  \acro{SPARQL}  {\acs{SPARQL} Protocol and RDF Query Language}
  \acro{SQL}     {Structured Query Language}

  \acro{TDB}     {Triple Data Base}
  \acro{TFM}     {Trabajo Fin de Máster}

  \acro{UCLM}    {Universidad de Castilla-La Mancha}
  \acro{UML}     {Unified Modeling Language}
  \acro{URI}     {Identificador Uniforme de Recurso}
  \acro{URL}     {Localizador Uniforme de Recurso}

  \acro{vCard}   {acs{vCard}}

  \acro{W3C}     {Consorcio para la World Wide Web}
  \acro{XML}     {Lenguaje de Marcas Extensible}

\end{acronym}
}


% \ac{OO}   la primera vez \acf, después \acs
% \acs{OO}  short: OO
% \acf{OO}  full : Object Oriented (OO)
% \acl{OO}  large: Object Oriented
% \acx{OO}         OO (Object Oriented)

% usa \Acro cuando no debe aparecer nunca expandido en el texto

% Local variables:
%   TeX-master: "main.tex"
% End:
